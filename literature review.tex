\documentclass{article}
\usepackage{times}
\usepackage{geometry}
\geometry{left=3.18cm, right=3.18cm, top=2.54cm, bottom=2.54cm}
\usepackage{indentfirst}
\usepackage[dvipsnames]{xcolor}
\usepackage{marginnote}
\begin{document}
\section{Searching results of researches on bus bunching problem}
\subsection*{Topic}
\noindent Bus bunching models and control schemes.

\subsection*{Databases consulted} 
\noindent The core collection of Web of Science;\\
Scopus.

\subsection*{Search strategy} 
\noindent Building block searching;\\
Cited reference search.

\subsection*{Used search terms}
\noindent TS=(bus bunching)\\
AND\\
SO=(transportation research part b methodological).

\subsection*{Notes}
\noindent 22 results.

\subsection*{Searching results of building block searching}

\begin{enumerate}
    \item 1980; Turnquist, M.A.a, Bowman, L.A.b\\
    The effects of network structure on reliability of transit service
    \item 1992; Adamski, A\\
    Probabilistic models of passengers service processes at bus stops
    \item 2009; Daganzo, Carlos F\\
    A headway-based approach to eliminate bus bunching: Systematic analysis and comparisons
    \item 2011; Daganzo, Carlos F\\
    Reducing bunching with bus-to-bus cooperation
    \item 2011; Xuan, Yiguang\\
    Dynamic bus holding strategies for schedule reliability: Optimal linear control and performance analysis
    \item 2011; Bartholdi, John J., III\\
    A self-coordinating bus route to resist bus bunching                                                                     
    \item 2012; Ibarra-Rojas, Omar J\\
    Synchronization of bus timetabling                                                                                       
    \item 2012; Delgado, Felipe\\
    How much can holding and/or limiting boarding improve transit performance?                                               
    \item 2015; Hernandez, Daniel\\
    Analysis of real-time control strategies in a corridor with multiple bus services                                        
    \item 2015; Argote-Cabanero, Juan\\
    Dynamic control of complex transit systems                                                                               
    \item 2015; Berrebi, Simon J\\
    % non
    A real-time bus dispatching policy to minimize passenger wait on a high frequency route                                  
    \item 2016; Liang, Shidong\\
    A self-adaptive method to equalize headways: Numerical analysis and comparison                                           
    \item 2016; Schmocker, Jan-Dirk\\
    Bus bunching along a corridor served by two lines                                                                        
    \item 2017; Wu, Weitiao\\
    Modelling bus bunching and holding control with vehicle overtaking and distributed passenger boarding behaviour          
    \item 2017; Andres, Matthias
    A predictive-control framework to address bus bunching                                                                   
    \item 2018; Zhang, Shuyang\\
    Two-way-looking self-equalizing headway control for bus operations                                                       
    \item 2018; Sirmatel, Isik Ilber\\
    Mixed logical dynamical modeling and hybrid model predictive control of public transport operations                      
    \item 2018; Petit, Antoine\\
    Dynamic bus substitution strategy for bunching intervention                                                              
    \item 2019; Li, Shukai\\
    Robust dynamic bus controls considering delay disturbances and passenger demand uncertainty                              
    \item 2019; Dai, Zhuang\\
    A predictive headway-based bus-holding strategy with dynamic control point selection: A cooperative game theory approach 
    \item 2019; Petit, Antoine\\
    Multiline Bus Bunching Control via Vehicle Substitution                                                                  
    \item 2020; Anderson, Paul\\
    Effect of transit signal priority on bus service reliability    
\end{enumerate}

\subsection*{Cited reference literature}
\noindent Authors (Year)[citations in Google Scholar, citations in Web of Science]\\

\noindent \textrm{Newell and Potts (1964)}[191,89]\\
\noindent \textrm{Potts and Tamlin (1964)}[23,9]\\
\noindent \textrm{Osuna and Newell (1972)}[429,216]\\
\noindent \textrm{Newell (1974)}[143,75]\\
\noindent \textrm{Barnett (1974)}[187,94]\\
\noindent \textrm{Newell (1977)}[?]\\
\noindent \textrm{Furth and Wilson (1981)}[238,117]\\
\noindent \textrm{Boyd (1983)}[25,8]\\
\noindent \textrm{Abkowitz and Lepofsky (1990)}[77,?]\\
\noindent \textrm{Ceder (1990)}[79,?]\\
\noindent \textrm{Wirasinghe and Liu (1995)}[47,29]\\
\noindent \textrm{Daganzo (1997a,b)}[a:4,3;b:16,?]\\
\noindent \textrm{Dessouky et al. (1999)}[202,121]\\
\noindent \textrm{Eberlein et al. (2001)}[279,158]\\
\noindent \textrm{Hickman (2001)}[234,141]\\
\noindent \textrm{Suh et al. (2002)}[59,?]\\
\noindent \textrm{Fu and Yang (2002)}[139,?]\\
\noindent \textrm{Dessouky et al. (2003)}[160,103]\\
\noindent \textrm{Liu et al. (2003)}[59,?]\\ 
\noindent \textrm{Kittelson (2003)}[35,?]\\
\noindent \textrm{Cathey and Dailey (2003)}[239,119]\\
\noindent \textrm{Fu et al. (2003)}[193,116]\\
\noindent \textrm{Jiang et al. (2003)}[32,27]\\
\noindent \textrm{Ling and Shalaby (2004)}[52,33]\\
\noindent \textrm{Sun and Hickman (2005)}[147,85]\\
\noindent \textrm{Shorter et al. (2005)}[93,61]\\
\noindent \textrm{Furth et al. (2006)}[117,?]\\
\noindent \textrm{Zhao et al. (2006)}[163,?]\\
\noindent \textrm{Puong and Wilson (2008)}[45,8]\\
\noindent \textrm{Daganzo et al. (2008)}[1001,?]\\
\noindent \textrm{Milkovits, (2008)}[2008,?]\\
\noindent \textrm{Hollander and Liu (2008)}[89,58]\\
\noindent \textrm{Sorratini et al. (2008)}[38,23]\\
\noindent \textrm{Fan and Machemehl (2009)}[67,38]\\
\noindent \textrm{Daganzo (2009a)}[359,225]\\ 
\noindent \textrm{Daganzo (2009b)}[10,?]\\
\noindent \textrm{Daganzo and Pilachowski (2009)}[252,164]\\ 
\noindent \textrm{Pilachowski et al. (2009)}[38,?]\\
\noindent \textrm{Delgado et al. (2009)}[142,86]\\
\noindent \textrm{Cortés et al. (2010)}[111,77]\\
\noindent \textrm{Cats et al. (2011)}[77,47]\\
\noindent \textrm{Xuan et al. (2011)}[210,147]\\
\noindent \textrm{Bartholdi and Eisenstein (2011)}[87,60]\\
\noindent \textrm{Daganzo and Pilachowski (2011)}[252,164]\\
\noindent \textrm{Feng and Figliozzi (2011a)}[22,?]\\
\noindent \textrm{Feng and Figliozzi (2011b)}[27,?]\\ 
\noindent \textrm{Bartholdi and Eisenstein (2012)}[197,133]\\
\noindent \textrm{Sáez et al. (2012)}[86,52]\\
\noindent \textrm{Moreira-Matias et al. (2012)}[31,14]\\
\noindent \textrm{Delgado et al. (2012)}[185,125]\\
\noindent \textrm{Muñoz et al. (2013)}[45,35]\\
\noindent \textrm{Liu et al. (2013)}[195,158]\\
\noindent \textrm{Nair et al. (2014)}[8,?]\\
\noindent \textrm{Moreira-Matias et al. (2014)}[24,10]\\
\noindent \textrm{Ibarra-Rojas et al. (2015)}[363,236]\\
\noindent \textrm{Argote-Cabanero et al. (2015)}[45,34]\\
\noindent \textrm{Hernandez et al. (2015)}[57,41]\\
\noindent \textrm{Hans et al. (2015)}[30,?]\\
\noindent \textrm{Berrebi et al. (2015)}[87,60]\\
\noindent \textrm{Fonzone et al. (2015)}[55,26]\\
\noindent \textrm{Chen et al. (2016)}[3,1]\\
\noindent \textrm{Sánchez-Martínez et al. (2016)}[73,45]\\
\noindent \textrm{Estrada et al. (2016)}[33,24]\\
\noindent \textrm{Zhao et al. (2016)}[10,6]\\
\noindent \textrm{Moreira-Matias et al. (2016)}[48,28]\\
\noindent \textrm{Verbich et al. (2016)}[19,10]\\
\noindent \textrm{Cats et al. (2016)}[77,47]\\
\noindent \textrm{Schmocker et al. (2016)}[39,26]\\
\noindent \textrm{Yu et al. (2016)}[41,34]\\
\noindent \textrm{Sun and Schmöcker (2017)}[12,8]\\
\noindent \textrm{Jiang et al. (2017)}[37,32]\\
\noindent \textrm{Estrada et al. (2017)}[33,24]\\
\noindent \textrm{Andres and Nair (2017)}[46,33]\\
\noindent \textrm{Wu et al. (2017)}[77,61]\\
\noindent \textrm{Alesiani and Gkiotsalitis (2018)}[3,2]\\
\noindent \textrm{Klumpenhouwer and Wirasinghe (2018)}[4,3]\\
\noindent \textrm{Ma et al. (2018a)}[51,50]\\
\noindent \textrm{Petit et al. (2018)}[74,17]\\
\noindent \textrm{Zhang et al. (2018)}[24,21]\\
\noindent \textrm{Varga et al. (2018)}[19,14]\\
\noindent \textrm{Berrebi et al. (2018)}[38,21]\\
\noindent \textrm{Manasra and Toledo (2019)}[4,3]\\
\noindent \textrm{Li et al. (2019)}[6,4]\\
\noindent \textrm{Dai et al. (2019)}[7,4]\\
\noindent \textrm{He et al. (2019)}[7,5]\\
\noindent \textrm{Petit et al. (2019)}[7,6]\\
\noindent \textrm{Wang et al. (2020)}[3,1]\\
\noindent \textrm{Anderson et al. (2020)}[11,2]\\

\section{Review of researches on bus bunching}

% What is the definition of bus bunching? (how many papers give definitions of bus bunching?)
\subsection*{Definition (What is bus bunching problem?)}
Bus bunching is a phenomenon stating that two or more buses 
arriving at stops with headways significantly shorter than designed.
This problem is caused by inevitable disturbances 
and enlarged by the natural characteristic of bus operation 
(Newell and Potts, 1964; Hickman, 2001; Daganzo, 2009)
and is prevalent especially in peak hours 
when passenger demand and service frequency are both high.\\

\noindent \textrm{Andres et al. (2017)}\\
One problem that comes along with highly frequent bus routes is called bus bunching, 
also known as platooning, 
which refers to the phenomenon when two or more buses of the same route arrive at the same time at a bus stop.\\

\noindent \textrm{Wu et al. (2017)}\\
In uncontrolled bus systems, 
bus bunching is prevalent especially in the peak hours.
% why in peak hour?
Bus bunching occurs when two or more buses along the same route 
arrive at a designated stop simultaneously.\\

\noindent \textrm{Petit et al. (2018)}\\
This phenomenon is well-known in the industry 
as an illustration of the instability of uncontrolled transit systems.
\textcolor{red}{\textrm{(Newell and Potts, 1964)}}\\

\noindent \textrm{Zhang et al. (2018)}\\
In public transport, 
bus bunching or bus platooning refers to the situation wherein two or more buses, 
which are supposed to be evenly spaced along the route, 
bunch together.\\

\noindent \textrm{Varga et al. (2018)}\\
This instability in public transport is called bus bunching
\textrm{(Pilachowski, 2009)}\\

\noindent \textrm{Dai et al. (2019)}\\
In public transport, 
bus bunching refers to the phenomenon 
when a group of two or more successive buses of a single line arrive at a stop 
with shorter headways than designed.\\

\noindent \textrm{He et al. (2019)}\\
The phenomenon that two or more buses servicing a bus line 
arrive at a bus stop at the same time or in a very short time interval 
is called bus bunching.\\

\noindent \textrm{Jiawei Wang (2020)}\\
In public transport systems, 
bus bunching refers to the phenomenon 
where a group of two or more buses arrives 
at the same bus stop at the same time.\\

\subsection*{Concerns (Why do we want to mitigate the bus bunching problem?)}
Bus bunching degrades the reliability and effectiveness of transit service
and concerns both passengers and transit agencies (Newell and Potts, 1964; Osuna and Newell, 1972; Hollander and Liu, 2008; Verbich et al. 2016).
From the perspective of passengers, the average waiting time at stops are increased and more passengers are likely to encounter overcrowded experiences onboard.
From the perspective of transit agencies, the underutilization of the pairing buses \textcolor{red}{raises} the operation cost to maintain the supply level,
i.e. transit agencies have to provide more services or bear the risk of losing split rate.\\

\noindent \textrm{Estrada et al. (2017)}\\
The reliability of transit modes is an important issue 
to ensure their competitiveness against the extended use of private cars in major cities.\\

\noindent \textrm{Andres et al. (2017)}\\
Bus bunching has negative effects for the passengers as well as for the service providers 
as it causes larger headways between separate bunches of buses. 
This leads on the one hand to longer waiting times for the passengers at the stops downstream 
and on the other hand to longer travel times for the passengers inside the bus 
due to longer boarding and dwell times at the stops.
The disadvantage for the service providers is that the costs are those of a highly frequent bus route, 
i.e., they need to provide many vehicles and drivers, 
but the real frequency experienced by the passengers is much smaller 
due to the larger headways between the separate bunches, 
which results in a low cost efficiency.\\

\noindent \textrm{Wu et al. (2017)}\\
The effectiveness of public transport system can be measured by its reliability.
This is undesirable for both passengers and transit operator 
since it leads to unexpectedly longer waiting times and degraded service reliability of public transport system 
\textcolor{red}{\textrm{(Hollander and Liu, 2008)}}.\\

\noindent \textrm{Petit et al. (2018)}\\
Reliability of service is a key performance indicator for transit agencies.
Schedule unreliability from bus bunching first affects passengers. 
As more passengers are served by late buses than by early buses, 
the expected waiting time for public transit passengers increases as the variance of the headways increases
\textcolor{red}{\textrm{(Osuna and Newell, 1972; Daganzo, 2008)}}.\\

\noindent \textrm{Zhang et al. (2018)}\\
Other than leading to irregular services and longer average waiting times for passengers, 
it will also result in underutilization of the buses bunched together, 
and consequential overcrowded services on the bus after the bunched buses.\\

\noindent \textrm{Berrebi et al. (2018)}\\
Bus bunching is the product of unstable dynamics that cause delays growing 
\textcolor{red}{\textrm{(Hickman, 2001)}}.
Unstable headway dynamics are a systemic problem that causes passenger wait and crowding.
% assumptions
\textrm{Fan and Machemehl (2009)} showed that on routes 
where headways are less than 12 min, 
passengers tend to arrive randomly.
Because more passengers arrive during long headways than during short ones, 
gaps in service cause disutility to passengers in the form of undue waiting time and crowding 
\textrm{(Newell and Potts, 1964; Milkovits, 2008; Cats et al., 2016)}.
Bus bunching also increases dwell time and running time, 
causing additional operating costs to the transit agency 
\textcolor{red}{\textrm{(Verbich et al. 2016)}}.\\

\noindent \textrm{Varga et al. (2018)}\\
At frequent lanes, 
if the schedule cannot be held and a bus arrives at the stop late, 
number of passengers is winding up.
It leads to non-homogeneous utilization of buses 
and therefore degradation of service quality.\\

\noindent \textrm{Dai et al. (2019)}\\
Resisting bus bunching to facilitate vehicle utilization 
and reduce passenger waiting time 
remains a long-standing challenge for transit agencies 
\textrm{(Ma et al., 2018a)}.\\

\noindent \textrm{Petit et al. (2019)}\\
This phenomenon is considered a plague in urban bus systems 
not only because it wreaks havoc on the punctuality and regularity of bus schedules but,
more importantly, 
because it imposes a heavy burden on the passengers’ travel experience 
(e.g., long waiting/dwelling times, crowded buses), 
which may in turn affect ridership and revenue.\\

\noindent Wu et al. (2019)\\
Liu and colleagues \textcolor{red}{(Liu and Sinha, 2007; Sorratini et al., 2008; Fonzone et al., 2015)} have shown that variable demand distribution, \marginpar[]{causes?}
both spatially and temporally, is the factor that affects bus reliability the most.\\

\noindent \textrm{Wang et al. (2020)}\\
Bus bunching affects service reliability and service efficiency.
It results in increased headway variability 
and longer waiting time at stops for passengers.
For transit agency, bus bunching causes ineffective use of the supply of public transport services 
and discourage the use of public transport.


\subsection*{Causes (Why bus bunching is inevitable and hard to deal with?)}
The bus bunching problem is prevalent in real life due to reasons of two aspects.
\begin{enumerate}
    \item The bus system suffers inevitable endogenous uncertainties and exogenous random disturbances.
    \item The natural characteristic of bus operation, a delayed bus picks more passengers while its following bus picks less, also enlarges the variability of bus headway.
\end{enumerate}
Under a conventional operation without countermeasures, a bus tends to pair with its preceding or following bus and can hardly recover to designed schedule or expected headway
especially when passenger demand and service frequency are both high.


The occurrence of bus bunching is mainly originated from the uncertainties of bus travel times on roads and bus dwell times at stops. 
On the one hand, bus travel times on roads are affected by some random issues such as urban traffic conditions, driver behaviors, etc.
On the other hand, fluctuate passenger demands also induce the uncertainties of bus dwell times at stops (Liu and Sinha, 2007; Sorratini et al., 2008; Fonzone et al., 2015).
Therefore, the resolve of bus bunching problem requires understanding and modeling both bus movements and passenger behaviors within bus operation. 
\textcolor{red}{Although$^2$} several bus bunching models/bus dynamic formulations/bus trajectory algorithms were proposed in literature to capture the propagation of bus bunching, 
most of the existing studies focused on oversimplified scenarios in which some critical issues like passenger transfer behavior and capacity constraint were not well addressed.\\

\noindent \textrm{Newell and Potts (1964)}\\
Even small disturbances when left uncontrolled 
can \textcolor{red}{grow rapidly} into large schedule deviations 
that appear equivalent to disruptions to the passengers.\\

\noindent \textrm{Liang et al. (2016)}\\
The disturbances in bus travel times and passenger arrivals at stops 
lead to failure in schedule adherence and headway uniformity.\\

\noindent \textrm{Andres et al. (2017)}\\
Based on a simplified model \textrm{Newell and Potts (1964)} proved the instability of a bus route,
i.e., the tendency of buses to pair together.
One of the main reasons for bus bunching is varying passenger boarding times 
and can be seen as a vicious cycle.
Based on an analysis of the bunching phenomenon and its causes 
\textrm{(Feng and Figliozzi, 2011b; Moreira-Matias et al., 2012; Verbich et al., 2016)}, 
and the understanding of the impact of a schedule based control 
\textcolor{red}{\textrm{(Newell, 1977; Zhao et al., 2006)}}. \marginpar[]{schedule based control}
\textrm{Feng and Figliozzi (2011a)} investigated the major causes for bus bunching.\\

\noindent \textrm{Wu et al. (2017)}\\
A series of factors contribute to bus bunching, 
such as stochastic running times and demand, vehicle capacity, driving manoeuvres, 
and passenger boarding behavior.
\textcolor{red}{\textrm{Fonzone et al. (2015)}} studied the impact of passengers’ timetable behaviour on bus bunching. 
They showed that the bus bunching phenomenon is in part due to such passengers’ timetable behavior.\\

\noindent \textrm{Petti et al. 2018}\\
Bus travel time is usually subject to randomness, 
e.g., buses have to travel within mixed traffic 
which is subject to congestion and the dwell time at a stop depends on the random number of boarding/alighting passengers, 
bus headways are likely to be irregular and unreliable.\\

\noindent \textrm{Zhang et al. (2018)}\\
\textrm{Fonzone et al. (2015)} discussed the relationship 
between passenger arrival patterns and dwell times that might lead to bunching.\\

\noindent \textrm{Berrebi et al. (2018)}\\
Even a small perturbation such as a traffic signal 
or a passenger paying in cash 
can destabilize the route and lead to bus bunching 
\textrm{(Kittelson, 2003; \textcolor{green}{Milkovits, 2008})}.\\

\noindent \textrm{Varga et al. (2018)}\\
Increased passenger demand and interactions with dense traffic 
are contributing factors to bus bunching.\\

\noindent \textrm{Li et al. (2019)}\\
Bus bunching is a common phenomenon especially along high-frequency bus lines
due to disturbances to bus running time 
and uncertainties in passenger arrival flow 
\textrm{(Osuna and Newell, 1972; Newell, 1974; Daganzo, 2009; 
\textcolor{red}{Daganzo and Pilachowski, 2011}; Fonzone et al., 2015; Schmocker et al., 2016)}.\\

\noindent \textrm{Dai et al. (2019)}\\
Bus bunching is inherently inevitable in transit operation 
due to the unstable collective bus motion. 
Despite having perfectly even departure frequencies, 
bus bunching can still occur as time elapses 
\textrm{(Daganzo, 2009)}. 
The reasons for bus bunching were first explained by \textrm{Newell and Potts (1964)} 
and have been expanded in other studies 
\textrm{(Barnett, 1974; Daganzo, 1997, 2009; Daganzo and Pilachowski, 2011)}.\\

\noindent \textrm{He et al. (2019)}\\
\textrm{Fonzone et al. (2015)} found the insufficient boarding rate 
and non-uniform arrival patterns may lead to severe bus bunching.\\

\noindent \textrm{Petit et al. (2019)}\\
An almost inevitable consequence is that bus headways and spacings become so irregular
that they would eventually end up “bunching” together 
\textrm{(Osuna and Newell, 1972; Daganzo, 2008)}\\

\noindent \textrm{Wang et al. (2020)}\\
Bus services are born unstable and in nature susceptible to bus bunching 
due to the inherent uncertainties in service operation. 
On the one hand, the uncertainty in dwelling time at stops due to time-varying passenger demand.
On the other hand, the uncertainty in travel time on roads due to time-varying traffic and driver conditions.
The delayed buses get even further delayed.\\

% what are the causes of bus bunching?
% are these two reasons mentioned by previous papers?

% review: what did they do? 
% what did they find? 
% what's the shortcomings of theri work?
\subsection*{Extensive efforts}
\subsubsection*{Detect/Prediction bus bunching}
\noindent \textrm{\textcolor{red}{Andres et al. (2017)}}\\
\textcolor{red}{A first general framework for the prediction of transit vehicle arrival and departure times based on AVL data }
has been presented by \textrm{\textcolor{red}{Cathey and Dailey (2003)}}, 
using a tracker, filter and predictor element.
Very few papers focus explicitly on the \textcolor{red}{prediction of bunching events} 
\textrm{(Nair et al., 2014; Moreira-Matias et al., 2014; 2016)}\\

\noindent \textrm{Zhang et al. (2018)}\\
\textrm{Andres and Nair (2017)} addressed bus bunching 
by considering both \textcolor{red}{data-driven headway prediction 
and dynamic holding strategies}.\\

\noindent \textrm{Berrebi et al. (2018)}\\
Several methods are based on predictions for the arrival times of \textcolor{red}{following buses}\marginpar[]{\textcolor{red}{why predict following buses?}} 
\textrm{(Bartholdi and Eisenstein, 2011; Daganzo and Pilachowski, 2011; Berrebi et al., 2015)}
using the prediction tool developed in \textrm{Hans et al. (2015)}.\\

\noindent \textrm{Varga et al. (2018)}\\
\textrm{Andres and Nair (2017)} used predictive algorithms 
to improve public transport reliability. 
Recent paper from \textrm{\textcolor{red}{Yu et al. (2016)}} employ already existing information 
to predict bus bunching employing information from transit smart cards.\\

\noindent \textrm{He et al. (2019)}\\
\textrm{Yu et al. (2016)} presented a \textcolor{red}{Support Vector Machine method 
to detect bus bunching with transit smart card data}.\\

\subsubsection*{Stop-Skipping}
\noindent \textrm{Petit et al. (2018)}\\
Recover from severe schedule disruptions 
\textrm{(\textcolor{red}{Fu et al., 2003; Sun and Hickman, 2005; Liu et al., 2013})}\\

\noindent\textrm{Dai et al. (2019)}\\
Short turning \textrm{(Fu et al., 2003; Zhang et al., 2017)}\\

\noindent \textrm{Petit et al. (2019), He et al. (2019), Wang et al. (2020)}\\
\textrm{(Suh et al., 2002; Fu et al., 2003; Sun and Hickman, 2005; Cortés et al., 2010; Liu et al. 2013)}\\

\noindent \textrm{Varga et al. (2018)}\\
\textrm{Jiang et al. (2017)} proposed a heuristic algorithm 
with stop skipping or inclusion for congested high-speed train lines.\\

\noindent Wu et al. (2019)\\
Limited-stop bus service (or a bus stop-skipping scheme) is one of the proven fleet allocation strategies to handle the unbalanced demand, 
which allows some vehicles to visit only a fixed subset of stops (Yu et al., 2012b; Liu et al., 2013).\\

\subsubsection*{Coordinated speed adjustment}
\noindent \textrm{He et al. (2019)}\\
\textrm{(Daganzo, 2009b; Daganzo and Pilachowski, 2009, 2011; 
He, 2015; Varga et al., 2018; Manasra and Toledo, 2019)}\\

\subsubsection*{Boarding restrictions/limits}
\noindent \textrm{Wang et al. (2020)}\\
Control bus dwell time
\textrm{(Osuna and Newell, 1972; Newell, 1974; Barnett, 1974; Delgado et al., 2009, 2012; Zhao et al., 2016)}\\

\subsubsection*{Traffic signal priority}
\noindent \textrm{Petit et al. 2018}\\
Consisting in optimizing the traffic flow 
(e.g., using controlled traffic lights) 
to help late buses to catch up 
but difficult to implement.
\textrm{(Liu et al., 2003; Ling and Shalaby, 2004; Estrada et al., 2016)}.\\

\noindent \textrm{Varga et al. (2018)}\\
A common method in improving timetable reliability provides priority to buses at signalized intersections.
\textrm{(Liu et al., 2003; Ling and Shalaby, 2004; Estrada et al., 2016)}\\

\subsubsection*{Substitution strategy}
\noindent \textrm{Petit et al. (2018)}\\

\subsubsection*{Overtaking}
\noindent \textrm{Zhang et al. (2018)}\\
\textrm{Sun and Schmöcker (2017)} modelled and explained passenger choices and overtaking
in the bus bunching problem.\\

\noindent \textrm{Wu et al. (2017)}\\


\subsection*{Shortcomings/drawbacks of these control strategies}
\noindent \textrm{Wu et al. (2017)}\\
Although stop-skipping scheme could increase the commercial speed, 
it also increases the waiting time of those passengers at the stops which are skipped.\\

\noindent \textrm{Wang et al. (2020)}\\
Stop-skipping and boarding restriction will penalize waiting passengers 
and they will make passengers even more unsatisfied, 
and other strategies essentially require additional resources/infrastructure 
from the agencies/operators. \\

\subsection*{Some methodologies}
\noindent \textrm{Wu et al. (2017)}\\
Most of the existing literature on bus propagation and holding control strategies 
presents simplified models 
without consideration of overtaking or passenger queue swapping behaviour.\\

\noindent \textrm{Zhang et al. (2018)}\\
Previous studies mainly focused on the development of a schedule 
\textrm{(Zhao et al., 2006; Xuan et al., 2011; He, 2015; Fonzone et al., 2015)} 
or a priori target headway 
\textrm{(Hickman, 2001; Fu and Yang, 2002; Daganzo, 2009; Daganzo and Pilachowski, 2011; Berrebi et al., 2015)}
to prevent bus bunching, 
and some strategies are based on rolling-horizon optimization
\textrm{(Eberlein et al., 2001; Delgado et al., 2009, 2012; Sáez et al., 2012; Sánchez-Martínez et al., 2016)}.\\

\noindent \textrm{Li et al. (2019)}\\
\textbf{Dynamic control methods} based on real-time information have been shown 
to alleviate these problems and improve resiliency alongside of schedule-improvement methods 
\textrm{(Daganzo, 2009; Muñoz et al., 2013; Sánchez-Martínez et al., 2016)}.
Dynamic control methods based on real-time feedback information
 have been proposed to alleviate these problems 
\textrm{(Eberlein et al., 2001; Dessouky et al., 2003; Shorter et al., 2005; 
Daganzo, 2009; Delgado et al., 2009; Hernandez et al., 2015; Andres et al., 2017)}.
Based on the real-time information,
\textrm{Eberlein et al. (2001)} proposed a \textbf{rolling horizon strategy} for the bus holding problem.\\

\subsection*{Holding control strategy}
\subsubsection*{Strength of bus holding strategy}
\noindent \textrm{Wang et al. (2020)}\\
Holding control (both static and dynamic) remains the most practical 
and most adopted strategy in real-world application. 
\textrm{(Cats et al., 2011; Wu et al., 2017b)}\\
Bus holding control strategies not only make passengers less frustrated 
but also provide better maneuverability to fleet operation.
\textrm{(Eberlein et al. 2001)}

\subsubsection*{Category of bus holding strategy}
\noindent \textrm{Wu et al. (2017)}\\
The holding controlling approaches can be classified into three groups: 
schedule-based control, headway-based control and optimisation-based control.
They are implemented through building slacks in the schedule at designated time points, 
in which the slacks are predetermined and static for schedule-based control 
while in headway- and optimisation-based holding strategies 
the slacks are determined in real-time.
Headway-based holding control approach is mainly triggered by headway deviation.
\textrm{Daganzo (2009)} explored a headway-based control scheme, 
in which the dynamic holding times are determined 
by taking advantage of the real-time forward headway information.
Optimisation-based models determine holding decisions through mathematical programming formulation,
with the objective to minimise the passenger waiting time or cost, 
either for the waiting passengers at-stops only or in combination with passengers in-vehicle.\\

\noindent \textrm{Petit et al. (2018)}\\
Holding strategies have been proposed to take advantage of \textcolor{red}{real-time information} \marginpar[]{\textcolor{red}{What is the real-time information for scheudle-based holding?}}
so as to reduce the waiting time at control points
\textrm{(\textcolor{red}{Abkowitz and Lepofsky, 1990}; Dessouky et al., 1999; \textcolor{red}{Hickman, 2001}; \textcolor{red}{Eberlein et al., 2001})}.
Headway-based dynamic holding strategies also used real-time information to develop adaptive control schemes 
\textrm{(\textcolor{red}{Daganzo, 2009}; \textcolor{red}{Bartholdi and Eisenstein}, 2012)}.
\textrm{\textcolor{blue}{Daganzo and Pilachowski (2011)}} used a two-way-looking speed control strategy, 
where buses cooperate with each other to eliminate bus bunching, 
even in case of large disruptions.
\textrm{\textcolor{red}{Xuan et al. (2011)}} expanded into robust versions 
that can deal with more complex systems; 
e.g., multiple interacting lines \textrm{\textcolor{red}{(Argote-Cabanero et al. 2015)}}\\

\noindent \textrm{Dai et al., (2019)}\\
\textrm{Wirasinghe and Liu (1995)} proposed an analytic model 
to determine the number, 
location, 
and slack time of control points for schedule-based bus holding.
\textrm{Klumpenhouwer and Wirasinghe (2018)} analyzed the optimal control point configuration
for schedule-based holding strategies.\\

\noindent \textrm{Petit et al. (2019)}\\
Early methods consist adding slacks into the schedule
\textrm{(\textcolor{red}{Newell, 1974; Abkowitz and Lepofsky, 1990; Eberlein et al., 2001})}.\marginnote[]{\textcolor{red}{adding slacks into schedules/static schedule-based holding}}[-1cm]
Recent holding strategies take advantage of real-time information 
and determine the holding time or the speed of buses adaptively  
\textrm{(Daganzo, 2009; Daganzo and Pilachowski, 2011; Xuan et al., 2011;
Bartholdi and Eisenstein, 2012; Berrebi et al., 2015; Sánchez-Martínez et al., 2016)}.\\

\noindent \textrm{He et al. (2019)}\\
Embedding-slack strategies 
\textrm{(Daganzo, 1997a,b; Zhao et al., 2006; Daganzo, 2009a; Xuan et al., 2011)}, 
Static and dynamic holding strategies 
\textrm{(Hickman, 2001; Eberlein et al., 2001, Sun and Hickman, 2008; 
Puong and Wilson, 2008; Daganzo, 2009a; Xuan et al., 2011; 
Bartholdi and Eisenstein, 2012; Delgado et al., 2012; He, 2015; Argote-Cabanero and Daganzo, 2015)}
The prediction-based methods outperform the methods without predictions \textrm{(Berrebi et al. 2018)}.\\

\noindent \textrm{Wang et al. (2020)}\\
Related studies on bus holding control can be mainly categorized into two classes: 
static headway/schedule-based control such as 
optimization schedule-based holding control by \textrm{Zhao et al. (2006)} 
and headway-based holding control by \textrm{Daganzo (2009)};
and dynamic control without a pre-specified headway or schedule
such as virtual schedule-based holding control by \textrm{Xuan et al. (2011)}.
DRL coordinative fleet control algorithms are proposed 
to address the bus bunching problem by 
\textrm{(Chen et al. 2016, Alesiani and Gkiotsalitis 2018, Wang et al. 2020.)}\\


\subsubsection*{Shortcomings/drawbacks of holding strategy}
\noindent \textrm{Wu et al. (2017)}\\
A drawback of holding control is that 
it may result in lengthened bus dwell times and overall travel time.\\

\noindent \textrm{Petit et al. (2018)}\\
Those holding strategies trade system stability for extra passenger dwell/travel time.\\

\noindent \textrm{Berrebi et al. (2018)}\\
There is a trade-off between stabilizing headways and maintaining high operating speed 
\textrm{(Furth et al., 2006; Furth and Wilson, 1981; Cats et al., 2011)}.
Several holding methods in the literature require setting a parameter, 
which affects the trade-off between holding time and headway stability 
\textrm{(Daganzo, 2009; Xuan et al., 2011; Bartholdi and Eisenstein, 2011; Daganzo and Pilachowski, 2011)}.
Several methods are based on predictions for the arrival times of following buses 
\textrm{(Bartholdi and Eisenstein, 2011; Daganzo and Pilachowski, 2011; Berrebi et al., 2015)}.


\section{Researches on bus bunching considering common line or bus capacity constraint}
\noindent Ibarra-Rojas et al. (2012)\\
We propose a formulation for the synchronization bus timetabling problem to reduce bus bunching 
and to optimize passenger transfer. 
Our formulation is based on Ceder et al. (2001), which was later improved by Eranki (2004).\\

\noindent Argote-Cabanero et al. (2015)\\
This paper proposes a dynamic control method to overcome bunching and improve the regularity of fixed-route transit systems.\\

\noindent Ibarra-Rojas et al. (2015)\\
Hall et al. (2001) develop analytical models that determine the optimal holding times at transfer stations with general bus arrival time distributions. 
The authors test the proposed approach on generated instances based on real data of the transit network of Los Angeles, CA.\\
Delgado et al. (2013) implement holding strategies to benefit passengers who must walk a little to transfer from line B to
another line A at a single stop.\\

\noindent Hernández et al. (2015)\\
This work develops an optimization model capable of executing a control scheme 
based on holding strategy for a corridor with multiple bus lines.
We analyzed the benefits in the level of service of the public transport system when considering a central operator 
who wants to maximize the level of service for users of all the bus lines, 
versus scenarios where each bus line operates independently.\\

\noindent \textrm{Andres et al. (2017)}\\
The problem becomes more complex, 
e.g., by considering multiple bus routes serving same stops 
\textrm{(Hernández et al., 2015; Schmöcker et al., 2016)}, 
or by considering capacity constraints 
\textrm{(Delgado et al., 2012; Jiang et al., 2003)}.\\

\noindent \textrm{Wu et al. (2017)}\\
Although there are many literature with various methods of operational control, to simplify the models, most of the
existing studies presented an oversimplified bus model, notably without overtaking.\\

\noindent \textrm{Zhang et al. (2018)}\\
Extra waiting time for passengers who cannot get on the first bus ($W_{extra}$),
 or waiting time at transfer ($W_{trans}$).\\

\noindent \textrm{Petit et al. (2018)}\\
While most relevant literature focused on a single transit line, 
efforts on multiline transit systems are relatively rare, 
and existing studies mainly address traditional control strategies.
\textrm{Hernández et al. (2015)} extended the work of \textrm{Delgado et al. (2012)} 
on limited-boarding and holding strategies to a system with multiple bus lines.
\textrm{Argote-Cabanero et al. (2015)} studied the adaptive control 
from \textrm{Xuan et al. (2011)} 
for multiline networks and presented a real-world application.\\

\noindent Dai et al. (2019)\\
It is important to note that holding control strategies are also used in the schedule coordination at transfer stations 
to synchronize bus arrivals and minimize passenger transfer waiting costs. 
For example, Sun and Schonfeld (2016) proposed a vehicle holding method 
that considers decision risks and correlations among vehicle arrivals to mitigate service disruptions. 
Wu et al. (2016) proposed a robust schedule coordination scheme which combines a delay control strategy and timetable planning. 
The delay control is realized by allowing holding for late buses within a safety control margin (SCM).

\section{Supports of assumptions}
\subsection*{Constant bus service frequency and passenger arrival rates} 
\noindent Wu et al. (2017)\\
Passenger arrivals at bus stops follow uniform distributions. 
This is a reasonable assumption for high frequency service, as validated and commonly used by many researchers 
(e.g. Salek and Machemehl, 1999; Sánchez-Martínez et al., 2016).\\

\noindent Wu et al. (2019)\\
A deterministic uniform distribution is commonly used to model the passenger arrival process in the literature, underlying that the demand variability is low.
However, urban transit demand usually exhibits high heterogeneity in both time and space, 
particularly in the high-density populated cities with large bus passenger demands (Yu et al., 2011).\\

\subsection*{Bus dwell time depended on the numbers of boarding and alighting passengers}
\noindent Wang et al. (2020)\\
The boarding and alighting processes start simultaneously and follow linear models of the number of passengers. 
The boarding time and alighting time per passenger are b and a, respectively, for all bus stops. 
Buses cannot leave the stop until both the boarding and the alighting processes finish, and dwell time is determined by the maximum of the two values. 
We assume that bus capacity is unlimited.\\

\noindent Li et al. (2019)\\
Dwell time is an affine function of the number of boarding passengers.\\

\subsection*{Constant travel time in the same links}
\noindent Schmöcker et al. (2016)\\
We assume that bus travel times between stops are constant and equal so that $v_{m(l)n}$ simplifies to $v$.\\

\noindent Hernández et al. (2015)\\
Travel times between stops and an estimation of all types of passengers’ arrival rates at each stop 
are assumed to be known and fixed over the period of interest.\\

\noindent Muñoz et al. (2013)\\
Passenger arrival rates for each stop and travel times between stops are deterministic, known and fixed over the period of interest.\\

\noindent Delgado et al. (2012)\\
Travel times between stops and an estimation of passenger arrival rates to each stop are assumed known and fixed over the period of interest.\\

\subsection*{Passenger OD demand information}
\noindent Wu et al. (2019)\\
The rapid development of information (i.e., the Internet of Things) enables the acquisition of real-time high-resolution data on passengers and vehicles. 
This facilitates transit operators to leverage on the potential flexibility embodied in limited-stop service in catering more efficiently 
to the prevailing passenger demand variations.\\

\section{Evaluation measurement and optimization objective}
Passengers' waiting time at stops and headway variability are commonly used as evaluation measurements and optimization objectives in related literature.\\

\noindent Delgado et al. (2012)\\
The objective function is to minimize the sum of the individual travel times of all passengers 
from the moment they arrive at a stop to the moment they reach their destination during the whole planning horizon. 
In this case the planning horizon consists in each vehicle visiting all stops exactly once. 
Since vehicle running times (while the vehicles are moving) are assumed to be constant, 
the objective is equivalent to minimizing waiting times (while vehicles are stopped) both in-vehicle and at-stops.\\

\noindent Hernández et al. (2015)\\
The objective function minimizes the average travel time of all passengers that travel from the moment they arrived at a
stop until they reach their destination during the planning horizon. 
The model just like the one from Delgado et al. (2012) assumes that travel time is constant, 
so the objective function only considers the waiting time at the stops and the waiting time in vehicle.\\

\noindent Liang et al. (2016)\\
The results include three parts: the standard deviations of headways, total waiting time of passengers, average bus travel
time.\\
\end{document}